\documentclass[lualatex]{jlreq}
\usepackage{physics2}
\usephysicsmodule{ab}
\usepackage{tikz}
\usepackage{pgfplots}
\pgfplotsset{compat=1.18}

\title{LaTeX to DOCX Converter サンプル}
\author{サンプル作成者}
\date{\today}

\begin{document}

\maketitle

\section{はじめに}

このドキュメントは、LaTeX to DOCX Converter の使用例を示しています。
カスタムコマンド、TikZ図、データプロットなど、様々な機能を含みます。

\section{カスタムコマンドの例}

physics2パッケージの \verb|\ab()| コマンドを使用した例:

\[
\ab(x + y) = \left( x + y \right)
\]

ネストされた括弧の例:

\[
\ab(\ab(a) + b) = \text{こういった式も対応します}
\]

\section{TikZ図の例}

\subsection{シンプルな幾何図形}

\begin{figure}[h]
\centering
\begin{tikzpicture}[scale=1.5]
    % 矩形
    \draw[thick] (0,0) rectangle (2,1);
    \node[anchor=center] at (1,0.5) {矩形};
    
    % 円
    \draw[thick, fill=lightgray] (4,0.5) circle (0.5);
    \node[anchor=center] at (4,0.5) {円};
    
    % 三角形
    \draw[thick] (6,0) -- (7,0) -- (6.5,1) -- cycle;
    \node[anchor=center] at (6.5,0.3) {三角形};
\end{tikzpicture}
\caption{基本的な幾何図形}
\label{fig:shapes}
\end{figure}

\subsection{データプロット}

\begin{figure}[h]
\centering
\begin{tikzpicture}
\begin{axis}[
    xlabel=X軸,
    ylabel=Y軸,
    width=8cm,
    height=6cm,
    grid=major,
    legend pos=north west
]
    \addplot[mark=*,blue] table {data/sample.dat};
    \addlegendentry{サンプルデータ}
\end{axis}
\end{tikzpicture}
\caption{データプロット例}
\label{fig:plot}
\end{figure}

\section{数式の例}

インライン数式の例:$E = mc^2$ は有名な式です。

ディスプレイ数式の例:

\begin{equation}
\frac{d}{dt}\ab(p(t)) = F(t)
\label{eq:newton}
\end{equation}

\section{結論}

このサンプルで示した機能が正常に DOCX ファイルに変換されることを確認してください。

\end{document}
